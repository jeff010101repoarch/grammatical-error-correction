\justifying
This project was about the tradition NLP task “Grammatical Error Correction”, aiming at developing an automatic system which can detect and correct the grammatical errors automatically. The objective is to use TensorFlow to build up and train a sequence to sequence model, that is capable of automatically correcting small grammatical errors in conversational written English, e.g., the messages on mobile phone. Instead of the traditional statistic model and manually rule design, we apply the latest powerful artificial intelligence tools, saying a specific RNN model with LSTM. The details of the configuration of neural network were provided. After building the neural network model, we also train the model on a GPU server and do performance test of it. Besides, we also explain how we overcome the lack of training dataset, i.e., we take English text samples which are known to be mostly grammatically correct and randomly introducing a handful of small grammatical errors to each sentence to produce input-output. As the results revealed, although sometimes the performance was not so good, this neural network model can detect and correct the usual typos which are made in the daily writing sentences. The feasibility of the Seq2seq model under LSTM suggested the promising future of applying artificial intelligence techniques in the grammatical error correction field.